\documentclass{article}
\usepackage[margin=0.7in]{geometry}
\usepackage[parfill]{parskip}
\usepackage[utf8]{inputenc}

\usepackage{amsmath,amssymb,amsfonts,amsthm}

\title{Discrete-Time Signal Processing Notes}
\author{Alan V. Oppenheim and Ronald W. Schafer as transcribed by poss}

\begin{document}
\maketitle{}

\section{Introduction}

DSP = good

\section{Discrete-Time Signals and Systems}

discrete time signals = sequence of number \(x\), in which the \(n\)th number in the sequence is denoted \(x[n]\)

\[
	x = {x[n]}, \quad n \in \mathbb{Z} 
\]

or more commonly and conveniently referenced as "the sequence \(x[n]\)"


some important sequences:


the \textit{unit sample sequence} is defined as the sequence
\[
	\delta{}[n] =
	\begin{cases} 
		0 & n \neq 0 \\
		1 & n = 0
	\end{cases}
\]

\end{document}
